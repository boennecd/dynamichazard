\RequirePackage{filecontents}

\begin{filecontents*}{mybib.bib}
@article{doucet09,
  title={A tutorial on particle filtering and smoothing: Fifteen years later},
  author={Doucet, Arnaud and Johansen, Adam M},
  journal={Handbook of nonlinear filtering},
  volume={12},
  number={656-704},
  pages={3},
  year={2009}
}

@article{kitagawa96,
  title={Monte Carlo filter and smoother for non-Gaussian nonlinear state space models},
  author={Kitagawa, Genshiro},
  journal={Journal of computational and graphical statistics},
  volume={5},
  number={1},
  pages={1--25},
  year={1996},
  publisher={Taylor \& Francis}
}

@inproceedings{douc05,
  title={Comparison of resampling schemes for particle filtering},
  author={Douc, Randal and Capp{\'e}, Olivier},
  booktitle={Image and Signal Processing and Analysis, 2005. ISPA 2005. Proceedings of the 4th International Symposium on},
  pages={64--69},
  year={2005},
  organization={IEEE}
}

@article{carpenter99,
  title={Improved particle filter for nonlinear problems},
  author={Carpenter, James and Clifford, Peter and Fearnhead, Paul},
  journal={IEE Proceedings-Radar, Sonar and Navigation},
  volume={146},
  number={1},
  pages={2--7},
  year={1999},
  publisher={IET}
}

@article{pitt99,
  title={Filtering via simulation: Auxiliary particle filters},
  author={Pitt, Michael K and Shephard, Neil},
  journal={Journal of the American statistical association},
  volume={94},
  number={446},
  pages={590--599},
  year={1999},
  publisher={Taylor \& Francis}
}

@incollection{pitt01,
  title={Auxiliary variable based particle filters},
  author={Pitt, Michael K and Shephard, Neil},
  booktitle={Sequential Monte Carlo methods in practice},
  pages={273--293},
  year={2001},
  publisher={Springer}
}

@article{andrieu02,
  title={Particle filtering for partially observed Gaussian state space models},
  author={Andrieu, Christophe and Doucet, Arnaud},
  journal={Journal of the Royal Statistical Society: Series B (Statistical Methodology)},
  volume={64},
  number={4},
  pages={827--836},
  year={2002},
  publisher={Wiley Online Library}
}

@article{fearnhead10,
  title={A sequential smoothing algorithm with linear computational cost},
  author={Fearnhead, Paul and Wyncoll, David and Tawn, Jonathan},
  journal={Biometrika},
  volume={97},
  number={2},
  pages={447--464},
  year={2010},
  publisher={Oxford University Press}
}

@inproceedings{briers05,
  title={Sequential auxiliary particle belief propagation},
  author={Briers, Mark and Doucet, Arnaud and Singh, Sumeetpal S},
  booktitle={Information Fusion, 2005 8th International Conference on},
  volume={1},
  pages={8--pp},
  year={2005},
  organization={IEEE}
}

@article{de97,
  title={The scan sampler for time series models},
  author={De Jong, Piet},
  journal={Biometrika},
  volume={84},
  number={4},
  pages={929--937},
  year={1997},
  publisher={Oxford University Press}
}

@article{briers10,
  title={Smoothing algorithms for state--space models},
  author={Briers, Mark and Doucet, Arnaud and Maskell, Simon},
  journal={Annals of the Institute of Statistical Mathematics},
  volume={62},
  number={1},
  pages={61--89},
  year={2010},
  publisher={Springer}
}
\end{filecontents*}

\documentclass{article}

\usepackage[round]{natbib}
\usepackage{amsmath} \usepackage{bm} \usepackage{amsfonts}
\usepackage{algorithm} \usepackage{algpseudocode} \usepackage{hyperref}

% algorithm and algpseudocode definitions
\newcommand\StateX{\Statex\hspace{\algorithmicindent}}
\newcommand\StateXX{\Statex\hspace{\algorithmicindent}\hspace{\algorithmicindent}}
\newcommand\StateXXX{\Statex\hspace{\algorithmicindent}\hspace{\algorithmicindent}\hspace{\algorithmicindent}}

\algnewcommand{\UntilElse}[2]{\Until{#1\ \algorithmicelse\ #2}}

\algtext*{EndFor} \algtext*{EndProcedure}

\algnewcommand\algorithmicinput{\textbf{Input:}}
\algnewcommand\INPUT{\item[\algorithmicinput]}

\newcommand{\citeAlgLine}[2]{line~\ref{#1} of algorithm~\ref{#2}}
\newcommand{\citeAlgLineTwo}[3]{line~\ref{#1} and~\ref{#2} of algorithm~\ref{#3}}
\newcommand{\citeAlgLineTo}[3]{lines~\ref{#1}-\ref{#2} of algorithm~\ref{#3}}

% Math commands
\renewcommand{\vec}[1]{\bm{#1}}
\newcommand{\vect}[1]{\widetilde{\vec{#1}}}
\newcommand{\vecb}[1]{\bar{\vec{#1}}}
\newcommand{\mat}[1]{\mathbf{#1}}
\newcommand{\matt}[1]{\widetilde{\mat{#1}}}

\newcommand{\Lbrace}[1]{\left\{ #1\right\}}
\newcommand{\Lparen}[1]{\left( #1\right)}
\newcommand{\Cond}[2]{\left. #1 \vphantom{#2} \right\vert  #2}

% Operators
\newcommand{\Prob}{\text{P}}
\newcommand{\VAR}{\text{Var}}
\newcommand{\E}{\text{E}}
\newcommand{\COV}{\text{E}}

\newcommand{\optor}[2]{#1\Lparen{#2}}
\newcommand{\optorC}[3]{\optor{#1}{\Cond{#2}{#3}}}

\newcommand{\prop}[1]{\optor{\Prob}{#1}}
\newcommand{\propC}[2]{\optorC{\Prob}{#1}{#2}}

\newcommand{\expec}[1]{\optor{\E}{#1}}
\newcommand{\expecC}[2]{\optorC{\E}{#1}{#2}}

\newcommand{\varp}[1]{\optor{\VAR}{#1}}
\newcommand{\varpC}[2]{\optorC{\VAR}{#1}{#2}}

\newcommand{\covp}[1]{\optor{\COV}{#1}}
\newcommand{\covpC}[2]{\optorC{\COV}{#1}{#2}}

\newcommand{\propAproxC}[2]{\optorC{\widetilde{p}}{#1}{#2}}

% Normal distribution
\newcommand{\normal}[2]{\optor{N}{#1,#2}}
\newcommand{\normalC}[3]{\optorC{N}{#1}{#2,#3}}

% ID: Importance density 
\newcommand{\IDC}[2]{\optorC{q}{#1}{#2}}
\newcommand{\IDAproxC}[2]{\optorC{\widetilde{q}}{#1}{#2}}

% Diagonal operator
\newcommand{\diag}[1]{\optor{\text{diag}}{#1}}

% KF notation
\newcommand{\KF}[3]{#1_{\left. #2 \right\vert #3}}
\newcommand{\KFSup}[4]{#1_{\left. #2 \right\vert #3}^{(#4)}}

% PF notation
\newcommand{\partic}[3]{#1_{#2}^{\Lparen{#3}}}
% B for backwards
\newcommand{\particB}[3]{\widetilde{#1}_{#2}^{\Lparen{#3}}}
% S for smoothed
\newcommand{\particS}[3]{\widehat{#1}_{#2}^{\Lparen{#3}}}

\newcommand{\bigO}[1]{\mathcal{O}\Lparen{#1}}

\begin{document}
This is note covers ideas for using a particle filter for the \verb|dynamichazard| package in R. See \url{https://cran.r-project.org/web/packages/dynamichazard/vignettes/ddhazard.pdf} for more information of the package. At this point, the estimation method includes the following (crude) approximations: an Extended Kalman filters, unscented Kalman filter and a mode approximations. You can run \verb|dynamichazard::ddhazard_app()| after installing the package to see the performance and computation time of the methods.

\section{Method}
\subsection*{Model}
The model is:

\begin{equation}
\begin{array}{ll}
 	y_{it} \sim \propC{y_{it}}{\theta_{it}} & \\
%
 	\vec{\theta}_{t} = \mat{X}_t\vec{\alpha}_t \\
% 
 	\vec{\alpha}_t = \mat{F}\vec{\alpha}_{t - 1} + \mat{R}\vec{\eta}_t &
 		\vec{\eta}_t \sim N\Lparen{\vec{0}, \mat{Q}} \\
%
	&	\vec{\alpha}_0 \sim N\Lparen{\vec{a}_0, \mat{Q}_0}
 \end{array}
\end{equation}%
%
where we denote the conditional densities as $\vec{y}_t \sim \optorC{g}{\cdot}{\vec{\alpha}_t}$ and $\vec{\alpha}_t \sim \optorC{f}{\cdot}{\vec{\alpha}_{t-1}}$. An alternative here could be to add noise to $\vec{\theta}_{t}$ and use the methods in \cite{andrieu02}.

For each $t$ we have risk set given by $R_t \subseteq \Lbrace{1,2,\dots,n}$. The observed outcomes are given by $\vec{y}_t = \Lbrace{y_{it}}_{i \in R_t}$. We are in a survival analysis setting where the simplest model has an indicator of death of individual $i$ in time $t$ such that $y_{it} \in \Lbrace{0, 1}$ where $\theta_{it}$ is the linear predictor where we use the logistic function as the link function. $\mat{X}_t$ is the matrix of the covariates and $\vec{\alpha}_t \in \mathbb{R}^p$ is the time-varying coefficients. The problems we are looking at have $n >> p$ (e.g. $n = 100000$ and $p = 20$). 

 I am planning to use a first order random walk for $\vec{\alpha}_t$ so I could drop $\mat{F}$. Further, I will need to estimate $\mat{Q}, \vec{\alpha}_0$ and fix $\mat{Q}_0$. To do so, I will use a particle filter to get smoothed estimates of $\vec{\alpha}_1, \dots, \vec{\alpha}_d$ and use an EM-algorithm.

To me, the most relevant smoothers seems to be those in \cite{fearnhead10} and \cite{briers10}. I have included the $\bigO{N}$ smoother in \cite{fearnhead10} shown in algorithm~\ref{alg:ONsmoother}. I will use this throughout the rest of this note.



\subsection*{Considerations}
Algorithm~\ref{alg:ONsmoother} requires that we specify the following importance densities and re-sampling weights (optimal values are given as the right hand side):

\begin{equation}\begin{matrix}
	\IDC{\vec{\alpha}_t}{\partic{\vec{\alpha}}{t-1}{j}, \vec{y}_t} = \propC{\vec{\alpha}_t}{\partic{\vec{\alpha}}{t-1}{j}, \vec{y}_t} \\
	\partic{\beta}{t}{j} \propto \propC{\vec{y}_t}{\partic{\vec{\alpha}}{t-1}{j}}\partic{w}{t-1}{j} \\
%
	\IDAproxC{\vec{\alpha}_t}{\vec{y}_t, \particB{\vec{\alpha}}{t + 1}{k}}\particB{\beta}{t}{k} \approx %
		\gamma_t\Lparen{\vec{\alpha}_t}
		\propC{\vec{y}_t}{\vec{\alpha}_t}
		\propC{\particB{\vec{\alpha}}{t + 1}{k}}{\vec{\alpha}_t}
		\frac{\particB{w}{t + 1}{k}}{\gamma_{t + 1}\Lparen{\particB{\vec{\alpha}}{t + 1}{k}}} \\
%
	\IDAproxC{\particS{\alpha}{t}{i}}{\partic{\vec{\alpha}}{t-1}{j},\vec{y}_t, \particB{\vec{\alpha}}{t+1}{k}} = 
		\propC{\vec{\alpha}_t}{\partic{\vec{\alpha}}{t-1}{j},\vec{y}_t, \particB{\vec{\alpha}}{t+1}{k}}
\end{matrix}\end{equation}

Further, we need to define a backwards filter distribution approximation:

\begin{equation}
	\propAproxC{\vec{\alpha}_t}{\vec{y}_{t:T}} \propto \gamma_t \Lparen{\vec{\alpha}_t}\propC{\vec{y}_{t:T}}{\vec{\alpha}_t}
\end{equation}% 
%
with an artificial prior distribution $\gamma_t \Lparen{\vec{\alpha}_t}$. See \cite{briers10} for how to select $\gamma_t$ and examples hereof. \cite{fearnhead10} also provides examples. 

Given the models of interest we have that:

\begin{itemize}
	\item Evaluating $\propC{\vec{y}_t}{\vec{\alpha}_t}$ is expensive operation as $n >> p$ and it has a $\bigO{np}$ computational cost. In general, any $\bigO{n^l}$ with $l \geq 1$ is going to take considerable time if it needed in any steps of the algorithms. 
	\item Evaluating $\propC{\vec{\alpha}_t}{\vec{\alpha}_{t -1}}$ and $\propC{\vec{\alpha}_t}{\vec{\alpha}_{t + 1}}$ is cheap and can be done in closed form.
	\item The second example in \cite{fearnhead10} is close to the model here though with $n = 1$ and $p = 2$. 
\end{itemize}

\subsubsection*{Forward filter (algorithm~\ref{alg:forward}}
Let $\normalC{\cdot}{\cdot}{\cdot}$ denote a (multivariate) normal distribution. Then we can select the proposal density as:
\begin{equation}
	\IDC{\vec{\alpha}_t}{\partic{\vec{\alpha}}{t-1}{j}, \vec{y}_t} = %
		\normalC{\vec{\alpha}_t}{\partic{\vec{\alpha}}{t-1}{j}}{\mat{Q}}
\end{equation}%
%
which we can sample from in $\bigO{p^2}$ time. Another option is to use normal approximation of $\vec{y}_t$:

\begin{equation}\begin{array}{cl}
	\optorC{g}{\vec{y}_t}{\vec{\alpha}_t} & \simeq  \optorC{\widetilde{g}_t}{\vec{y}_t}{\vec{\alpha}_t} \\ &= %
		\normalC{
			\mat{X}_t\vec{\alpha}_t
		}{ % end of arg1 \normalC
			\mat{X}_t\mat{F}\vecb{\alpha}_{t - 1} - 
			\mat{G}_t\Lparen{\mat{F}\vecb{\alpha}_{t - 1}}^{-1}
			\vec{g}_t\Lparen{\mat{F}\vecb{\alpha}_{t - 1}}		
		}{ % end of arg2 \normalC
			- \mat{G}_t\Lparen{\mat{F}\vecb{\alpha}_{t - 1}}^{-1}
		}
\end{array}\end{equation}
%
% 
where:
\begin{equation}\begin{array}{c}
\mat{g}_t\Lparen{\vec{\alpha}} = 
		\Lbrace{\left.\frac{
		\partial \log \propC{y_{it}}{\theta_{it}}
	}{
		\partial\theta_{it}
	}\right|_{\theta_{it} = \vec{x}_{it}^\top\vec{\alpha}} }_{i \in R_t} \\
%
	\mat{G}_t\Lparen{\vec{\alpha}} = 
		\diag{\Lbrace{\left.\frac{
		\partial^2 \log \propC{y_{it}}{\theta_{it}}
	}{
		\partial\theta_{it}^2
	}\right|_{\theta_{it} = \vec{x}_{it}^\top\vec{\alpha}} }_{i \in R_t}}
\end{array}\end{equation}
%
%
to get:
\begin{equation}
	\IDC{\vec{\alpha}_t}{\partic{\vec{\alpha}}{t-1}{j}, \vec{y}_t} \propto %
		\normalC{\vec{\alpha}_t}{\mat{F}\partic{\vec{\alpha}}{t-1}{j}}{\mat{Q}}
		\optorC{\widetilde{g}}{\vec{y}_t}{\mat{F}\partic{\vec{\alpha}}{t-1}{j}}
\end{equation}
%
%
where we can analytically integrate out $\widetilde{g}$ to get:
\begin{equation}\begin{array}{rcl}
	\IDC{\vec{\alpha}_t}{\partic{\vec{\alpha}}{t-1}{j}, \vec{y}_t} &=&  %
		\normalC{\vec{\alpha}_t}{\vec{\mu}_t}{\mat{\Sigma}_t} \\
%
	\mat{\Sigma}_t^{-1} &=& \mat{X}_t^\top 
		\Lparen{-\mat{G}_t\Lparen{\mat{F}\vecb{\alpha}_{t - 1}}}
		\mat{X}_t + \mat{Q}^{-1}  \\
%
	\vec{\mu}_t &=& \mat{\Sigma}_t\Lparen{
		\mat{Q}^{-1}\mat{F}\partic{\vec{\alpha}}{t-1}{j} + 
		\mat{X}_t^\top \Lparen{-\mat{G}_t\Lparen{\mat{F}\vecb{\alpha}_{t - 1}}}
		\Lparen{
			\mat{X}_t\mat{F}\vecb{\alpha}_{t - 1} - 
			\mat{G}_t\Lparen{\mat{F}\vecb{\alpha}_{t - 1}}^{-1}
			\vec{g}_t\Lparen{\mat{F}\vecb{\alpha}_{t - 1}}
		}
	} \\
%
	 &=& \mat{\Sigma}_t\Lparen{
		\mat{Q}^{-1}\mat{F}\partic{\vec{\alpha}}{t-1}{j} + 
		\mat{X}_t^\top
		\Lparen{
			\Lparen{-\mat{G}_t\Lparen{\mat{F}\vecb{\alpha}_{t - 1}}}
			\mat{X}_t\mat{F}\vecb{\alpha}_{t - 1} +
			\vec{g}_t\Lparen{\mat{F}\vecb{\alpha}_{t - 1}}
		}
	}
\end{array}\end{equation}
%
%

We can select  $\vecb{\alpha}_{t - 1}$ as the mode or mean given the cloud at time $t-1$. This is similar to the second order random walk example in \cite{fearnhead10}. The downside is an $\bigO{np^2}$ computational cost though independent of the number of particles. Next, we have the re-sampling weights. A simple solution is not to use an auxiliary particle filter and set:

\begin{equation}
	\partic{\beta}{t}{j} \propto \partic{w}{t-1}{j}
\end{equation}
%
which is an $\bigO{1}$ operation. Another options is to set:

\begin{equation}
	\partic{\beta}{t}{j} \propto  \optorC{g}{\vec{y}_t}{\mat{F}\partic{\vec{\alpha}}{t-1}{j}}\partic{w}{t-1}{j}
\end{equation}
%
%
which is an $\bigO{np}$ operation per particle.

The computational cost of re-weighting the particles in~\eqref{eqn:forwardWeight} depend on the above choices. It is at least an $\bigO{np}$ operation due to $\propC{\vec{y}_t}{\partic{\vec{\alpha}}{t}{i}}$.

\subsubsection*{Backward filter (algorithm~\ref{alg:backward}}
We need to specify $\gamma_t\Lparen{\vec{\alpha}_t}$. \citet[page 69 and 70]{briers10} provides recommendation on the selection. This leads to:
\begin{equation}\begin{matrix}
	\gamma_t\Lparen{\vec{\alpha}_t} =
		\normalC{\vec{\alpha}_t}{\vec{a}_0}{\mat{P}_t} \\
%
	\mat{P}_t = \left\{
		\begin{matrix} \mat{Q}_0 & t = 0 \\ \mat{F}\mat{P}_{t - 1}\mat{F}^\top + \mat{Q} & t > 0   \end{matrix} \right.
\end{matrix}\end{equation}

The considerations regarding $\IDAproxC{\vec{\alpha}_t}{\vec{y}_t, \particB{\vec{\alpha}}{t + 1}{k}}$ and $\particB{\beta}{t}{k}$ are similar to those of the forward filter. We can select the importance density as:

\begin{equation}\begin{array}{cl}
	\IDAproxC{\vec{\alpha}_t}{\vec{y}_t, \particB{\vec{\alpha}}{t + 1}{k}}  &= %
		\propC{\particB{\vec{\alpha}}{t + 1}{k}}{\vec{\alpha}_t}
		\frac{\gamma_t\Lparen{\vec{\alpha}_t}}{\gamma_{t + 1}\Lparen{\particB{\vec{\alpha}}{t + 1}{k}}} \\
	& = \normalC{\particB{\vec{\alpha}}{t + 1}{k}}{\vec{\alpha}_t}{\mat{Q}}
		\frac{
			\normalC{\vec{\alpha}_t}{\vec{a}_0}{\mat{P}_t}
		}{ % ednd \frac arg1 
			\normalC{\particB{\vec{\alpha}}{t + 1}{k}}{\vec{a}_0}{\mat{P}_{t + 1}}
		}
\end{array}\end{equation}

We can add a normal approximation of $\propC{\vec{y}_t}{\vec{\alpha}_t}$ factor as with the forward filter. This again leads to an $\bigO{np^2}$ computational cost. The considerations regarding the re-sampling weights, $\particB{\beta}{t}{k}$, are very close to those of $\partic{\beta}{t}{k}$.

\subsubsection*{Combining / smoothing (algorithm~\ref{alg:ONsmoother}}
We need to specify the importance density $\IDAproxC{\cdot}{\partic{\vec{\alpha}}{t-1}{j},\vec{y}_t, \particB{\vec{\alpha}}{t+1}{k}}$ (see \citet[page 453] {fearnhead10}). We can select:

\begin{equation}
	\IDAproxC{\vec{\alpha}_t}{\partic{\vec{\alpha}}{t-1}{j},\vec{y}_t, \particB{\vec{\alpha}}{t+1}{k}} \propto %
%
	\propC{\vec{\alpha}_t}{\partic{\vec{\alpha}}{t - 1}{j}}
	\frac{
		\propC{\particB{\vec{\alpha}}{t + 1}{k}}{\vec{\alpha}_t}
	}{
		\gamma_{t + 1}\Lparen{\particB{\vec{\alpha}}{t + 1}{k}}
	}
\end{equation}

Again, we can add a normal approximation of the $\propC{\vec{y}_t}{\vec{\alpha}_t}$ factor.

\subsubsection*{Conclusion}
Add the least, all methods will end up being $\bigO{np}$ due to computation of the weights in equation~\eqref{eqn:combineWeight},~\eqref{eqn:forwardWeight} and~\eqref{eqn:backwardWeight} as we end up with $2N + N_s$ evaluations of $\propC{\vec{y}_t}{\vec{\alpha}_t}$.

We can add (crude) estimate for the $\propC{\vec{y}_t}{\vec{\alpha}_{t - 1}}$ and $\propC{\vec{y}_t}{\vec{\alpha}_{t + 1}}$ at  $2N$ evaluations of  $\propC{\vec{y}_t}{\vec{\alpha}_t}$.

We can make a normal approximation of  $\propC{\vec{y}_t}{\vec{\alpha}_t}$ in any of the importance densities but this will give at an $\bigO{np^2}$ cost per particle.

We can do as in the second example in \citet[page~462 and 463]{fearnhead10} to get an algorithm with $\bigO{d\Lparen{\Lparen{N + N_s}np + np^2}}$ computational cost.

\newpage

TODO: Change some P's to f's and g's?

TODO: add what to do on time 0 and $d$.

\begin{algorithm}
\caption{$\bigO{N}$ two filter smoother using the method in \cite{fearnhead10}.}\label{alg:ONsmoother}
\begin{algorithmic}[1]\raggedright
\INPUT
\Statex $\mat{Q},\mat{Q}_0,\vec{a}_0,\mat{X}_1,\dots,\mat{X}_d,\vec{y}_1,\dots,\vec{y}_d,R_1,\dots,R_d,\mat{S}$
%
\Statex Let $\partic{\vec{\alpha}}{t}{i}$ denote particle $i$ at time $t$, $\partic{w}{t}{i}$ denote the weight of the particle and $\partic{\beta}{t}{i}$ denote the re-sampling weight.
%
\Statex Importance density which optimally is (see \citet[page 453]{fearnhead10}): 
\Statex \begin{equation}
	\IDAproxC{\vec{\alpha}_t}{\partic{\vec{\alpha}}{t-1}{j},\vec{y}_t, \particB{\vec{\alpha}}{t+1}{k}} = 
%
	\propC{\vec{\alpha}_t}{\partic{\vec{\alpha}}{t-1}{j},\vec{y}_t, \particB{\vec{\alpha}}{t+1}{k}}
\end{equation}
%
\Procedure{Filter forward}{}
\State Run a forward particle filter to get a particle swarm % 
	$\Lbrace{\partic{\vec{\alpha}}{t}{j}, \partic{w}{t}{j}, \partic{\beta}{t}{j}}_{j=1,\dots,N}$ %
	approximating $\propC{\vec{\alpha}_t}{\vec{y}_{1:t}}$ for $t = 0, 1, \dots, d$. See algorithm~\ref{alg:forward}.
\EndProcedure
%
\Procedure{Filter backwards}{}
\State Run a similar backward filter to get %
	$\Lbrace{\particB{\vec{\alpha}}{t}{k}, \particB{w}{t}{k}, \particB{\beta}{t}{k}}_{k=1,\dots,N}$  %
	approximating $\propC{\vec{\alpha}_t}{\vec{y}_{t:d}}$ for $t = d, d-1, \dots , 2$. See algorithm~\ref{alg:backward}.
\EndProcedure
% 
\Procedure{Smooth (combine)}{}
\For{$t=1,\dots, d - 1$}
\StateXX \emph{Re-sample:}
\State $i=1,2,\dots,N_s$ pairs of $\Lparen{j_i, k_i}$ where each component is independently sampled using re-sampling weights $\partic{\beta}{t}{j}$ and $\particB{\beta}{t}{k}$
%
\StateXX \emph{Propagate:} 
\State Sample particles $\particS{\vec{\alpha}}{t}{i}$ from importance density %
	$\IDAproxC{\cdot}{\partic{\vec{\alpha}}{t-1}{j_i},\vec{y}_t, \particB{\vec{\alpha}}{t + 1}{k_i}}$
.%
\StateXX \emph{Re-weight:}
\State Assign each particle weights:
\StateXX \begin{equation}\label{eqn:combineWeight}
 \particS{w}{t}{i} \propto \frac{
 	\propC{\particS{\vec{\alpha}}{t}{i}}{\partic{\vec{\alpha}}{t - 1}{j_i}}
 	\propC{\vec{y}_t}{\particS{\vec{\alpha}}{t}{i}}
 	\propC{\particB{\vec{\alpha}}{t + 1}{k_i}}{\particS{\vec{\alpha}}{t}{i}}
 	\partic{w}{t - 1}{j_i}\particB{w}{t + 1}{k_i}
 	}{ % end of first argument of \frac
 	\IDAproxC{\particS{\vec{\alpha}}{t}{i}}{\partic{\vec{\alpha}}{t-1}{j_i},\vec{y}_t, \particB{\vec{\alpha}}{t+1}{k_i}}
 	\partic{\beta}{t}{j_i}\particB{\beta}{t}{k_i}\gamma_{t +1}\Lparen{\particB{\vec{\alpha}}{t+1}{k_i}}
 	} % end of frac
\end{equation}
\EndFor
\EndProcedure
\end{algorithmic}
\end{algorithm}



\begin{algorithm}
\caption{Forward filter due to \cite{pitt99}. You can compare with \citet[page 20 and 25]{doucet09}. The version and notation below is from \citet[page 449]{fearnhead10}.}\label{alg:forward}
\begin{algorithmic}[1]\raggedright
\INPUT
\State Importance density and specification of weights are optimally given by: 
\Statex \begin{equation}\begin{matrix}
	\IDC{\vec{\alpha}_t}{\partic{\vec{\alpha}}{t-1}{j}, \vec{y}_t} = \propC{\vec{\alpha}_t}{\partic{\vec{\alpha}}{t-1}{j}, \vec{y}_t} \\
	\partic{\beta}{t}{j} \propto \propC{\vec{y}_t}{\partic{\vec{\alpha}}{t-1}{j}}\partic{w}{t-1}{j}
\end{matrix}\end{equation}
\For{$t=1,\dots, d$}
\Procedure{Resample}{}
\State Compute re-sampling weights $\partic{\beta}{t}{j}$ and re-sample according to $\partic{\beta}{t}{j}$ to get indices $j_1,\dots j_N$.
\EndProcedure
%
\Procedure{Propagate}{}
\State Sample new particles $\partic{\vec{\alpha}}{t}{i}$ using importance density $\IDC{\vec{\alpha}_t}{\partic{\vec{\alpha}}{t-1}{j_i}, \vec{y}_t}$.
\EndProcedure
%
\Procedure{Re-weight}{}
\State Re-weight particles using:
\StateX \begin{equation}\label{eqn:forwardWeight}
	\partic{w}{t}{i} \propto \frac{
		\propC{\vec{y}_t}{\partic{\vec{\alpha}}{t}{i}}
		\propC{\partic{\vec{\alpha}}{t}{i}}{\partic{\vec{\alpha}}{t-1}{j_i}}
		\partic{w}{t-1}{j_i}
	}{ % \frag arg1 end
		\IDC{\partic{\vec{\alpha}}{t}{i}}{\partic{\vec{\alpha}}{t-1}{j_i}, \vec{y}_t}
		\partic{\beta}{t}{j_i}
	}
\end{equation}
\EndProcedure
\EndFor
\end{algorithmic}
\end{algorithm}



\begin{algorithm}
\caption{Backwards filter.}\label{alg:backward}
\begin{algorithmic}[1]\raggedright
\INPUT
\Statex A backwards filter distribution approximation:
\begin{equation}
	\propAproxC{\vec{\alpha}_t}{\vec{y}_{t:T}} \propto \gamma_t \Lparen{\vec{\alpha}_t}\propC{\vec{y}_{t:T}}{\vec{\alpha}_t}
\end{equation} 
\Statex with an  artificial prior distribution $\gamma_t \Lparen{\vec{\alpha}_t}$. This is introduced as $\propC{\vec{y}_{t:T}}{\vec{\alpha}_t}$ is not a density function in $\vec{\alpha}_t$.
\Statex Importance density and specification of weights {\footnotesize (\citet[page 451  -- maybe look in the example in the appendix]{fearnhead10})}:
\Statex\begin{equation}
	\IDAproxC{\vec{\alpha}_t}{\vec{y}_t, \particB{\vec{\alpha}}{t + 1}{k}}\particB{\beta}{t}{k} \approx %
		\gamma_t\Lparen{\vec{\alpha}_t}
		\propC{\vec{y}_t}{\vec{\alpha}_t}
		\propC{\particB{\vec{\alpha}}{t + 1}{k}}{\vec{\alpha}_t}
		\frac{\particB{w}{t + 1}{k}}{\gamma_{t + 1}\Lparen{\particB{\vec{\alpha}}{t + 1}{k}}}
\end{equation}
\Statex where I figure that we want {\footnotesize (the first follows from \citet[page 74]{briers10})}:
\Statex\begin{equation}\begin{matrix}
	\IDAproxC{\vec{\alpha}_t}{\vec{y}_t, \particB{\vec{\alpha}}{t + 1}{k}} \propto %
		\propC{\vec{y}_t}{\vec{\alpha}_t}
		\propC{\particB{\vec{\alpha}}{t + 1}{k}}{\vec{\alpha}_t}
		\frac{\gamma_t\Lparen{\vec{\alpha}_t}}{\gamma_{t + 1}\Lparen{\particB{\vec{\alpha}}{t + 1}{k}}} \\
	\particB{\beta}{t}{k} \propto %
		 \propAproxC{\vec{y}_t}{\particB{\vec{\alpha}}{t + 1}{k}}\particB{w}{t + 1}{k}
\end{matrix}\end{equation}
\For{$t=d - 1,\dots, 1$}
\Procedure{Resample}{}
\State Compute re-sampling weights $\particB{\beta}{t}{k}$ and re-sample according to $\particB{\beta}{t}{k}$ to get indices $k_1,\dots k_N$.
\EndProcedure
%
\Procedure{Propagate}{}
\State Sample new particles $\partic{\vec{\alpha}}{t}{i}$ using importance density $\IDAproxC{\vec{\alpha}_t}{\particB{\vec{\alpha}}{t + 1}{k_i}, \vec{y}_t}$.
\EndProcedure
%
\Procedure{Re-weight}{}
\State Re-weight particles using {\footnotesize (see \citet[page 72]{briers10} and add the ratios of weights and re-sampling weights)}:
\StateX \begin{equation}\label{eqn:backwardWeight}
	\particB{w}{t}{i} \propto \frac{
		\propC{\vec{y}_t}{\partic{\vec{\alpha}}{t}{i}}
		\propC{\partic{\vec{\alpha}}{t + 1}{k_i}}{\partic{\vec{\alpha}}{t}{i}}
		\partic{w}{t + 1}{k_i}
		\gamma_t\Lparen{\partic{\vec{\alpha}}{t}{i}}
	}{ % \frag arg1 end
		\IDC{\partic{\vec{\alpha}}{t}{i}}{\partic{\vec{\alpha}}{t + 1}{k_i}, \vec{y}_t}
		\partic{\beta}{t}{k_i}
		\gamma_{t + 1}\Lparen{\partic{\vec{\alpha}}{t + 1}{k_i}}
	}
\end{equation}
\EndProcedure
\EndFor
\end{algorithmic}
\end{algorithm}






\begin{equation}\begin{array}{c}
	\matt{F} = \begin{pmatrix}
		\mat{R} & \mat{0} & \mat{0} \\
		\mat{F}\mat{R} &\mat{R} & \mat{0} \\
		\mat{F}^2\mat{R} & \mat{F}\mat{R} & \mat{R}
	\end{pmatrix} \\
%
	\matt{Q} = \matt{F}\begin{pmatrix}
		\mat{Q} & \mat{0} & \mat{0} \\
		\mat{0} & \mat{Q} &  \mat{0} \\
		\mat{0} & \mat{0} & \mat{Q}
	\end{pmatrix}\matt{F}^\top \\
%
	\vect{a}_0 = \begin{pmatrix}
		\expec{\vec{a}_{t - 1}} \\ \mat{F}\expec{\vec{a}_{t - 1}} \\ \mat{F}^{2}\expec{\vec{a}_{t - 1}}
	\end{pmatrix} = 
	\begin{pmatrix}
		\mat{F}^{-2}\expec{\vec{a}_{t + 1}} \\ \mat{F}^{-1}\expec{\vec{a}_{t + 1}} \\ \expec{\vec{a}_{t + 1}}
	\end{pmatrix} \\
%
	\begin{pmatrix}
		\vec{\alpha}_{t-1} \\ \vec{\alpha}_{t} \\ \vec{\alpha}_{t+1}
	\end{pmatrix} \sim \normal{\vect{a}_0}{\matt{Q}}
\end{array}\end{equation}

\begin{equation}\begin{array}{cl}
	\expecC{\vec{\alpha}_t}{\vec{\alpha}_{1:(t - 1)}, \vec{\alpha}_{(t + 1):d}} \hspace{-50pt}& \\ 
%
 	& =  \Lparen{\vect{a}_0}_2 + \covp{\Lparen{\vect{a}_0}_2, \Lparen{\vect{a}_0}_{\{1, 3\}}} % 
 		\varp{\Lparen{\Lparen{\vect{a}_0}_{1, 3}}}^{-1} %
 		\Lparen{\begin{pmatrix} \vec{\alpha}_{t - 1} \\ \vec{\alpha}_{t + 1}\end{pmatrix} - \Lparen{\vect{a}_0}_{\{1, 3\}}}
\end{array}\end{equation}











\section{Literature review}
\subsubsection*{\cite{doucet09}}
\begin{itemize}
	\item Do re-sample and likely use \emph{Systematic Resampling} from \cite{kitagawa96}. If you use it, then read up on \cite{douc05}.
	\item Use threshold on \emph{effective sample size} to deside when to re-sample.
	\item Use \emph{Auxiliary Particle Filtering}. That is, use the information from the $t + 1$ outcome when re-computing the weights (\textit{sample before re-sampling}). See \cite{carpenter99}, \cite{pitt01} and \cite{pitt99}.
	\item Likely use MCMC Moves (re-sample moves) or block sampling limit degeneracy problem. See text for references.
	\item Rao-Blackwellised Particle Filtering: always exploit when you can have closed form solutions like when parts of the problem is (conditional) multivariate Gaussian. See the text and \cite{andrieu02} for examples. 
	\item Two filter smother is likely the way to go. It may be better `\textit{the support of the smoothed estimate differs substantially from that of the filtering estimate}`. The cost is $\bigO{N^2T}$ but can be reduced to $\bigO{NT}$ using the methods in \cite{fearnhead10} and \cite{briers05}.
\end{itemize}

\subsubsection*{\cite{andrieu02}}
Covers a Rao–Blackwellized particle filtering where the state is linear and Gaussian and observations are (potentially) non-linear and non-Gaussian. 

\begin{itemize}
	\item The filter exploits closed form solutions from Gaussian part.
	\item It is not an auxiliary particle filter so w likely want to change this.
	\item Has a low memory requirement as we do not need to store a smooth covariance matrix for each particle.
	\item Mentions a MCMC move approach for sampling to deal with degeneracy. See \cite{de97}.
\end{itemize}

\subsubsection*{\cite{fearnhead10}}
Covers a two filter smoother:
\begin{itemize}
	\item Run an \emph{auxiliary particle filtering} forward and backward filter.
	\item The backwards filter uses samples from `\textit{marginal smoothing densities direct than re-weight those drawn from another distribution}`. This solves degeneracy problems  like those that occurs for a second order random walk. They use that the forward and backwards sets of particle clouds (or swarms) are independent. Then we sample in the combination step using particles from $t-1$ from the forward filter and particles from $t + 1$ from the backwards filter. The computational cost in $\bigO{N^2T}$.
	\item Get an $\bigO{NT}$ filter by sampling a forward particle and backward particle pair at each time $t$ while combining.
	\item Shows examples which are fully or partially linear Gaussian models. Derivations are in the appendix.
	\item Points out in the discussion that we likely want more particles in smoothing/combination step than in the forward and backwards filters.
\end{itemize}



\iffalse
Our model is:

\begin{equation}
\begin{array}{ll}
 	y_{it} \sim \propC{y_{it}}{\theta_{it}} & \\
%
 	\vec{\theta}_{t} = \mat{X}_t\vec{\alpha}_t + \vec{\epsilon}_t & 
  		\vec{\epsilon}_t \sim N\Lparen{\vec{0}, \mat{G}_t} \\
% 
 	\vec{\alpha}_t = \mat{F}\vec{\alpha}_{t - 1} + \mat{R}\vec{\eta}_t &
 		\vec{\eta}_t \sim N\Lparen{\vec{0}, \mat{Q}} \\
%
	&	\vec{\alpha}_0 \sim N\Lparen{\vec{a}_0, \mat{Q}_0}
 \end{array}
\end{equation}

where $y_{it}$ and $y_{jt}$ with $i\neq j$ are conditionally independent given $\theta_{it}$ and $\theta_{jt}$. This is the setup in \cite{andrieu02}. For each $t$ we have risk set given by $R_t \subseteq \Lbrace{1,2,\dots,n}$.

Next, we define:

\begin{equation}
\begin{matrix}
\KF{\vec{a}}{t}{s} = \expecC{\vec{\alpha}_t}{\vec{\theta}_{s}} \\
\KF{\mat{P}}{t}{s} = \varpC{\vec{\alpha}_t}{\vec{\theta}_{s}} \\
%
\KF{\vec{t}}{t}{s} = \expecC{\vec{\theta}_t}{\vec{\theta}_{s}} \\
\KF{\mat{S}}{t}{s} = \varpC{\vec{\alpha}_t}{\vec{\theta}_{s}}
\end{matrix}
\end{equation}

where it follows from Standard Kalman filtering that: 
\begin{equation}
\begin{matrix}
\KF{\vec{a}}{t}{1:(t - 1)} = \mat{F}\KF{\vec{a}}{t - 1}{1:(t - 1)} \\
\KF{\mat{P}}{t}{1:(t - 1)} = \mat{F}\KF{\mat{P}}{t - 1}{1:(t - 1)}\mat{F}^\top + \mat{Q} \\
\KF{\vec{\theta}}{t}{1:(t - 1)} = \mat{X}_t \KF{\vec{a}}{t}{1:(t - 1)} \\
\KF{\mat{S}}{t}{1:(t - 1)} = \mat{X}_t \KF{\mat{P}}{t}{1:(t - 1)} \mat{X}_t^\top + \mat{G}_t \\
%
\KF{\vec{a}}{t}{1:t} = \KF{\vec{a}}{t}{1:(t - 1)} + \KF{\mat{P}}{t}{1:(t - 1)} \mat{X}_t^\top %
	\KF{\mat{S}}{t}{1:(t - 1)}^{-1}\Lparen{ \vec{\theta}_t - \KF{\vec{\theta}}{t}{1:(t - 1)} } \\
\KF{\mat{P}}{t}{1:t} = \KF{\mat{P}}{t}{1:(t - 1)} - \KF{\mat{P}}{t}{1:(t - 1)} \mat{X}_t^\top \KF{\mat{S}}{t}{1:(t - 1)}^{-1} %
	\mat{X}_t\KF{\mat{P}}{t}{1:(t - 1)}
\end{matrix}
\end{equation}

We will use the forward filter in \cite{andrieu02} in we add superscripts fot he particle index and the above formulas become:
\fi

\newpage
\bibliographystyle{plainnat}
\bibliography{mybib}

\end{document} 